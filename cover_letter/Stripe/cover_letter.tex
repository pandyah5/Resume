%%%%%%%%%%%%%%%%%%%%%%%%%%%%%%%%%%%%%%%%%
% Friggeri Cover Letter
% XeLaTeX Template
% Version 1.1 (24/12/18)
%
% This template has been downloaded from:
% https://github.com/mlda065/friggeri-letter
%
% Original author:
% Matthew Davis, based on code by
% Adrien Friggeri (adrien@friggeri.net)
% https://github.com/afriggeri/CV
%
% License:
% License:
%    Copyright (C) 2012, Adrien Friggeri
%    Permission is hereby granted, free of charge, to any person obtaining a copy of this software and associated documentation files (the "Software"), to deal in the Software without restriction, including without limitation the rights to use, copy, modify, merge, publish, distribute, sublicense, and/or sell copies of the Software, and to permit persons to whom the Software is furnished to do so, subject to the following conditions:
%    The above copyright notice and this permission notice shall be included in all copies or substantial portions of the Software.
%    THE SOFTWARE IS PROVIDED "AS IS", WITHOUT WARRANTY OF ANY KIND, EXPRESS OR IMPLIED, INCLUDING BUT NOT LIMITED TO THE WARRANTIES OF MERCHANTABILITY, FITNESS FOR A PARTICULAR PURPOSE AND NONINFRINGEMENT. IN NO EVENT SHALL THE AUTHORS OR COPYRIGHT HOLDERS BE LIABLE FOR ANY CLAIM, DAMAGES OR OTHER LIABILITY, WHETHER IN AN ACTION OF CONTRACT, TORT OR OTHERWISE, ARISING FROM, OUT OF OR IN CONNECTION WITH THE SOFTWARE OR THE USE OR OTHER DEALINGS IN THE SOFTWARE.
%
% Important notes:
% This template needs to be compiled with XeLaTeX
% You may need to compile twice for the header to appear.
%
%%%%%%%%%%%%%%%%%%%%%%%%%%%%%%%%%%%%%%%%%

\documentclass[a4paper,english]{friggeri-letter}

\usepackage{babel}

\begin{document}

\header{Hetav~}{Pandya}{Computer Engineering (AI Minor) at University of Toronto}

\address{
   85 Wood St \\
   Toronto, ON, Canada
}


\letter{
   354 Oyster Point Blvd, United States
   % 19 Allstate Parkway \\
   % Markham, ON L3R 5A4
}



\opening{Dear Recruiter at Stripe:}

\textbf{About Me}

I recently graduated with a degree in computer engineering with a minor in artificial intelligence at the University of Toronto (UofT). I am a person who likes to build and learn from the experience. My courses have helped me challenge my assumptions to explore other interesting approaches to the problem at hand. My passion lies in building software solutions and in the process I ended up learning a lot of technologies and skills. My primary area of interest within software lies in Machine Learning and Data Science.

% \textbf{Why Stripe?}


\textbf{Why Stripe?}

There are two major reasons why I believe Stripe is a good fit for my working style and career goals. My educational background is heavily skewed towards technology as is obvious from my resume, however, lately I have developed a keen interest in finance, so much so that I have read nine financial systems and literacy books in the last 12 months. Secondly, I do relate heavily with many of Stripe's operating principles. 
I have been to several hackathons which have taught me to deliver sophisticated projects in a very short time frame. Moving with urgency and focus is a requirement for most of the competitions I have participated in. 
Furthermore, I treat my craft very seriously. May it be academics, hackathons, or leadership roles; I have always worked on spending every passing minute on learning skills to improve my performance. 
For example, when I was first elected as a Co-President for my club, I voluntarily enrolled in a leadership course at University and read books to better handle crucial conversations. 
In my internships, I have made it a habit to go a step beyond and engage myself in courses and seminars that help me upskill myself. Stagnation, not failure is what I fear.


% My courses at the UofT, have helped me develop a strong base in Python, C++, C and computer architecture. I have also invested a lot of time going beyond my university and have completed 11 online courses related to ML and data science. I took these courses to help me build many projects which later helped me win hackathons. They gave me the competitive experience to learn on the go and produce good projects in a short time frame. Additionally, my experience at Intel's and Bell's Big Data and AI team helped me understand the resource requirements and methodologies of AI in big industries. In my role as the Co-President of the UofT Machine Intelligence Student Team (UTMIST), I have closely supervised many applied and research projects surrounding ML along with hosting some several skill development workshops.

\vspace*{0.1cm}
\closing{
   Yours Sincerely\\
   Hetav Pandya}

\end{document}
