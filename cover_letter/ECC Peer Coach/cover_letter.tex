%%%%%%%%%%%%%%%%%%%%%%%%%%%%%%%%%%%%%%%%%
% Friggeri Cover Letter
% XeLaTeX Template
% Version 1.1 (24/12/18)
%
% This template has been downloaded from:
% https://github.com/mlda065/friggeri-letter
%
% Original author:
% Matthew Davis, based on code by
% Adrien Friggeri (adrien@friggeri.net)
% https://github.com/afriggeri/CV
%
% License:
% License:
%    Copyright (C) 2012, Adrien Friggeri
%    Permission is hereby granted, free of charge, to any person obtaining a copy of this software and associated documentation files (the "Software"), to deal in the Software without restriction, including without limitation the rights to use, copy, modify, merge, publish, distribute, sublicense, and/or sell copies of the Software, and to permit persons to whom the Software is furnished to do so, subject to the following conditions:
%    The above copyright notice and this permission notice shall be included in all copies or substantial portions of the Software.
%    THE SOFTWARE IS PROVIDED "AS IS", WITHOUT WARRANTY OF ANY KIND, EXPRESS OR IMPLIED, INCLUDING BUT NOT LIMITED TO THE WARRANTIES OF MERCHANTABILITY, FITNESS FOR A PARTICULAR PURPOSE AND NONINFRINGEMENT. IN NO EVENT SHALL THE AUTHORS OR COPYRIGHT HOLDERS BE LIABLE FOR ANY CLAIM, DAMAGES OR OTHER LIABILITY, WHETHER IN AN ACTION OF CONTRACT, TORT OR OTHERWISE, ARISING FROM, OUT OF OR IN CONNECTION WITH THE SOFTWARE OR THE USE OR OTHER DEALINGS IN THE SOFTWARE.
%
% Important notes:
% This template needs to be compiled with XeLaTeX
% You may need to compile twice for the header to appear.
%
%%%%%%%%%%%%%%%%%%%%%%%%%%%%%%%%%%%%%%%%%

\documentclass[a4paper,english]{friggeri-letter}

\usepackage{babel}

\begin{document}

\header{Hetav~}{Pandya}{Computer Engineering (AI Minor) at University of Toronto}

% \address{
%    85 Wood St \\
%    Toronto, ON, Canada
% }


\letter{
   255 Beverley Street, Toronto, CA 
}

\opening{Dear Recruiter at ECC:}

\textbf{About Me}

I am a fourth year computer engineering student pursuing a minor in artificial intelligence at the University of Toronto (UofT).
As I progress into my last year at UofT, I have come to value the various opportunities that have been given to me at this institution. 
One such was the PEY Co-op experience. I have participated in many different internships before but having a 12-month long program gave me a new perspective into the corporate work culture.

\textbf{Why a PEY Peer Coach?}

The equation is simple, the PEY program helped me and hence I am looking for ways to return the favor. 
I recently had an opportunity to give back when I had a chat with Sonja, an SDO at ECC, about how I could contribute to the upcoming Edge conference.
The conference presents me an opportunity to cover the fundamental learnings from my 12 month experience keeping in mind the general audience.
However, I believe there is a lot more specific help I could provide if I could connect with people individualy in their PEY preperation journey. 
The role of a PEY Peer Coach, provides me with exactly this opportunity!

I remember the days, when I used to do midnight resume reviews and mock interview sessions for my friends. 
Because for many of us, PEY provided the first shot at a professional work experience, a chance to prove ourselves.
Having gone through this process, I believe I can help in such a role! 

\textbf{Why Me?}

Since my freshman year, I have tried to increase my knowledge and experience base, beyond the academic curriculum. 
Over the years, this attitude helped me work in a diverse spread of industries including Education, Automobile, Telecommunications and Semi-conductors.
I have taught STEM to high school students during my time at Engineering Outreach. 
I gained valuable insights in the automobile industry at General Motors.
I was fortunate enough to experience the telecommunication and semi-conductor industries when working for Bell and Intel respectively.
Like many at UofT, I had a deep inclination towards research and I have spent a semester working with the Faculty of Information on a research project.

These experiences have shaped my outlook on career growth. 
If I get a chance to work as a Peer Coach, I would love to use my experience to help students make the best decisions for their careers!


\vspace*{0.1cm}
\closing{
   Yours Sincerely\\
   Hetav Pandya}

\end{document}
