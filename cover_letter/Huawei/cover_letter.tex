%%%%%%%%%%%%%%%%%%%%%%%%%%%%%%%%%%%%%%%%%
% Friggeri Cover Letter
% XeLaTeX Template
% Version 1.1 (24/12/18)
%
% This template has been downloaded from:
% https://github.com/mlda065/friggeri-letter
%
% Original author:
% Matthew Davis, based on code by
% Adrien Friggeri (adrien@friggeri.net)
% https://github.com/afriggeri/CV
%
% License:
% License:
%    Copyright (C) 2012, Adrien Friggeri
%    Permission is hereby granted, free of charge, to any person obtaining a copy of this software and associated documentation files (the "Software"), to deal in the Software without restriction, including without limitation the rights to use, copy, modify, merge, publish, distribute, sublicense, and/or sell copies of the Software, and to permit persons to whom the Software is furnished to do so, subject to the following conditions:
%    The above copyright notice and this permission notice shall be included in all copies or substantial portions of the Software.
%    THE SOFTWARE IS PROVIDED "AS IS", WITHOUT WARRANTY OF ANY KIND, EXPRESS OR IMPLIED, INCLUDING BUT NOT LIMITED TO THE WARRANTIES OF MERCHANTABILITY, FITNESS FOR A PARTICULAR PURPOSE AND NONINFRINGEMENT. IN NO EVENT SHALL THE AUTHORS OR COPYRIGHT HOLDERS BE LIABLE FOR ANY CLAIM, DAMAGES OR OTHER LIABILITY, WHETHER IN AN ACTION OF CONTRACT, TORT OR OTHERWISE, ARISING FROM, OUT OF OR IN CONNECTION WITH THE SOFTWARE OR THE USE OR OTHER DEALINGS IN THE SOFTWARE.
%
% Important notes:
% This template needs to be compiled with XeLaTeX
% You may need to compile twice for the header to appear.
%
%%%%%%%%%%%%%%%%%%%%%%%%%%%%%%%%%%%%%%%%%

\documentclass[a4paper,english]{friggeri-letter}

\usepackage{babel}

\begin{document}

\header{Hetav~}{Pandya}{Computer Engineering (AI Minor) at University of Toronto}

\address{
   310 Bloor St W \\
   Toronto, ON, Canada
}


\letter{
   19 Allstate Parkway \\
   Markham, ON L3R 5A4
}



\opening{Dear Recruiter at Huawei:}

\textbf{About Me}

I am a third year computer engineering student pursuing a minor in artificial intelligence at the University of Toronto (UofT). I am a person who likes to build and learn from the experience. My courses have helped me challenge my assumptions to explore other interesting approaches to the problem at hand. My passion lies in building software solutions and in the process I ended up learning a lot of technologies and skills. My primary area of interest within software lies in Machine Learning and Data Science.

\textbf{Why Huawei?}

I wanted to gain experience from an organization that values innovation and implements it in its very roots. Huawei has one of the largest patent portfolios in the world, with 100,000+ active patents. A company where 105,000 employees work in R\&D, is what impressed me and attracted me to Huawei. Coming from a research-intensive university and having served as a research assistant myself, I realize how important innovation is for the future. Based on Huawei's current focus on research, there's no doubt that it has the potential to be the next big thing in the future. I wish to be a part of that future!

\textbf{Why Me?}

My courses at the UofT, have helped me develop a strong base in C, C++, object-oriented programming, and computer architecture. I have also invested a lot of time going beyond my university and have completed 11 online courses related to ML and data science. I took these courses to help me build many projects which later helped me win hackathons. They gave me the competitive experience to learn on the go and produce good projects in a short time frame. Additionally, my experience at Bell's Big Data and AI team helped me understand the resource requirements and methodologies of AI in big industries. In my role as the Co-President of the UofT Machine Intelligence Student Team (UTMIST), I have closely supervised many applied and research projects surrounding ML along with hosting some skill development workshops.

\vspace*{0.1cm}
\closing{
   Yours Sincerely\\
   Hetav Pandya}

\end{document}
