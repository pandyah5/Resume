\documentclass[a4paper]{article}
    \usepackage{fullpage}
    \usepackage{hyperref}
    \usepackage{amsmath}
    \usepackage{amssymb}
    \usepackage{textcomp}
    \usepackage[utf8]{inputenc}
    \usepackage[T1]{fontenc}
    \textheight=10in
    \pagestyle{empty}
    \raggedright

    %\renewcommand{\encodingdefault}{cg}
%\renewcommand{\rmdefault}{lgrcmr}

\def\bull{\vrule height 0.8ex width .7ex depth -.1ex }

% DEFINITIONS FOR RESUME %%%%%%%%%%%%%%%%%%%%%%%

\newcommand{\area} [2] {
    \vspace*{-9pt}
    \begin{verse}
        \textbf{#1}   #2
    \end{verse}
}

\newcommand{\lineunder} {
    \vspace*{-8pt} \\
    \hspace*{-18pt} \hrulefill \\
}

\newcommand{\header} [1] {
    {\hspace*{-18pt}\vspace*{6pt} \textsc{#1}}
    \vspace*{-6pt} \lineunder
}

\newcommand{\employer} [3] {
    { \textbf{#1} (#2)\\ \underline{\textbf{\emph{#3}}}\\  }
}

\newcommand{\contact} [3] {
    \vspace*{-10pt}
    \begin{center}
        {\Huge \scshape {#1}}\\
        #2 \\ #3
    \end{center}
    \vspace*{-8pt}
}

\newenvironment{achievements}{
    \begin{list}
        {$\bullet$}{\topsep 0pt \itemsep -2pt}}{\vspace*{4pt}
    \end{list}
}

\newcommand{\schoolwithcourses} [4] {
    \textbf{#1} #2 $\bullet$ #3\\
    #4 \\
    \vspace*{5pt}
}

\newcommand{\school} [4] {
    \textbf{#1} #2 $\bullet$ #3\\
    #4 \\
}
% END RESUME DEFINITIONS %%%%%%%%%%%%%%%%%%%%%%%

    \begin{document}
\vspace*{-40pt}



%==== Profile ====%
\vspace*{-10pt}
\begin{center}
	{\Huge \scshape {Hetav Pandya}}\\
	Toronto, CA $\cdot$ \href{mailto:pandyahetav1@gmail.com}{pandyahetav1@gmail.com} $\cdot$ (416) 826-4057 $\cdot$ \href{https://www.linkedin.com/in/hetav-pandya}{LinkedIn} $\cdot$ \href{https://github.com/pandyah5}{GitHub} $\cdot$ \href{https://pandyah5.github.io/terminalresume.github.io/}{Personal Website}\\
\end{center}

%==== Education ====%
\header{Education}
\textbf{University of Toronto - GPA: 3.97}\hfill Toronto, CA\\
High Honors in Computer Engineering with Artificial Intelligence Minor \hfill Sep 2019 - May 2024\\
Cumulative average: 92.5\%, ranked \#2 in UofT Engineering in 2019-20
\linebreak\\
\begin{footnotesize}
\textit{Linear Algebra: 99/100 | Algorithms and DS: 89/100 | Operating Systems: 89/100 | Electronics: 99/100 | 
 Intro. to Machine Learning: 95/100 | Computer Networks: 90/100 | Software Engineering: 90/100 | Databases: 95/100}
\end{footnotesize}
\vspace{2mm}

%==== Skills ====%
\header{Skills}
\begin{tabular}{ l l }
	Programming Languages: & Python, C++, C, R, Go, Kotlin, JavaScript, Perl, TCL, Bash             \\
	Machine Learning:      & TensorFlow, PyTorch, AWS SageMaker, Ollama, Docker, Kubeflow             \\
	Data Analysis:         & PostgreSQL, MongoDB, MySQL, Database Design, Power BI            \\
	Productivity Tools:    & JIRA, Confluence, Docker, Git, GitHub, Perforce           \\
    Additional skills:     & Flask, RestAPI, Verilog, ARM Assembly, FPGA Hardware, Linux                      \\                               
\end{tabular}
\vspace{2mm}

%==== Experience ====%
\header{Work Experience}
\vspace{1mm}

\textbf{Dept. of Mathematics, University of Toronto} \hfill Toronto, CA\\
\textit{Teaching Assistant - TA} \hfill September 2023 - April 2024\\
\vspace{-3mm}
\begin{itemize} \itemsep 0.5pt
	\item Taught \textbf{Linear Algebra, Fundamental Calculus and Differential Equations} to engineering students at the University of Toronto.
\end{itemize}

\textbf{Intel Corp.} \hfill Toronto, Canada\\
\textit{Software Engineering Intern} \hfill May 2022 - May 2023\\
\vspace{-3mm}
\begin{itemize} \itemsep 1pt
	\item Developed a netlist writer in \textbf{C++} for Quartus Prime 2023 Q2 release.
	\item Implemented IP pin-mapping to reduce time taken to compile customer designs by 15\%.
	\item Used profiling tools like \textbf{VTune and flamegraphs} to optimize code and identify bottlenecks. 
	\item Worked with the router team to setup formal verification of our top model Agilex \textbf{FPGAs}. 
\end{itemize}

\textbf{Bell Enterprises} \hfill Toronto, Canada\\
\textit{Data Scientist Intern} \hfill May 2021 - Aug 2021\\
\vspace{-3mm}
\begin{itemize} \itemsep 1pt
	\item Met with stakeholders to determine bottlenecks in performance. 
	\item Optimized \textbf{MySQL} data queries reducing the time taken by 60\% on average.
	\item Developed an automated production deep learning pipeline in \textbf{Python and Kubeflow} to detect potential flaws in new version releases, reducing detection time from 3 weeks to 15 minutes.
\end{itemize}

\textbf{General Motors (GM)} \hfill Toronto, Canada\\
\textit{Machine Learning Model Developer} \hfill May 2021 - July 2021\\
\vspace{-3mm}
\begin{itemize} \itemsep 1pt
	\item Worked on automating data collection pipeline with data pre-processing and image augmentation.
	\item Deployed a real-time custom \textbf{object detection model} with mean Average Precision of 0.93.
\end{itemize}

\textbf{University of Toronto} \hfill Toronto, Canada\\
\textit{Data Analyst Research Intern - Faculty of Information} \hfill Jan 2021 - May 2021\\
\vspace{-3mm}
\begin{itemize} \itemsep 1pt
    \item Analyzed the effects of machine learning on the future path of job creation and disruption.
	\item Used \textbf{Python (Selenium, Beautiful Soup)} to retrive and visualize data from multiple sources.
\end{itemize}

% \textbf{Engineering Outreach, University of Toronto} \hfill Toronto, CA\\
% \textit{Software Content Specialist} \hfill Feb 2020 - May 2020\\
% \vspace{-1mm}
% \begin{itemize} \itemsep 1pt
% 	\item Designed the “Computer Imaging using Python” and the “Data Analytics using MATLAB” courses.
% 	\item The courses were made for the DEEP program, which is an educational program offered by UofT.
% \end{itemize}

\header{Projects}
{\textbf{Nash Equilibria convergence using RL}} {\sl RLib, DQN, PPO, Python} $\cdot$ \href{https://github.com/BoundlessDevelopment/Capstone-Project}{View Project}\\
Worked with Prof. Lacra Pavel, to develop Reinforcement Learning algorithms that converge to Nash Equilibria in partial information networks
 in autonomous drone networks and smart power grids.\\
\vspace*{2mm}
{\textbf{Open Hansard}} {\sl Llama3, Ollama, Python} $\cdot$ \href{https://github.com/pandyah5/open-hansard}{View Project}\\
Open Hansard is an open-source initiative to summarize the debates of the canadian parliament using state-of-the-art open-source LLMs.\\
\vspace*{2mm}
{\textbf{PEY Door}} {\sl Claude LLM, AWS SageMaker, Web Dev} $\cdot$ \href{https://devpost.com/software/peydoor}{View Project}\\
A first-place winner in AWS Student Hack, it used a Claude LLM in AWS Sagemaker to answer student's PEY related questions from the official internship reports submitted by the UofT students.\\
\vspace*{2mm}
{\textbf{Toronto Armour}} {\sl Kotlin, Jetpack Compose, Android} $\cdot$ \href{https://github.com/pandyah5/TorontoArmour}{View Project} \\
An open-source android application that alerts the users when they enter neighbourhoods with high safety risks based on official Toronto police data.\\
\vspace*{2mm}
{\textbf{E-Motion}} {\sl Python, OpenCV, Selenium} $\cdot$ \href{https://devpost.com/software/e-motion-otbl2i}{View Project} \\
A computer vision suite that enables users to play games, read e-books and listen to music using hand gestures. It secured the second place at UofT Hacks VIII\\
\vspace*{2mm}
{\textbf{ECE-Hustler}} {\sl C language, ARM Assembly, DE1-SoC board} $\cdot$ \href{https://github.com/pandyah5/ECE-Hustler}{View Project} \\
In this game you are an ECE student in the 2nd year at UofT trying to dodge the hurdles we faced! It is an obstacle course compiled on our custom-built ARM processor and displayed on a VGA display.\\
\vspace*{2mm}
{\textbf{Asphalt 9 Hands-Free Simulator}} {\sl Python, OpenCV} $\cdot$ \href{https://github.com/pandyah5/asphalt9_OpenCV_Simulation?tab=readme-ov-file}{View Project} \\
This project is about creating a gaming interface that allows the user to control and play games solely using hand gestures. A demo test was done on the game - Asphalt 9\\
\vspace*{2mm}
% {\textbf{Drowsy Driver Detector}} {\sl Python, OpenCV, Face Detection} \hfill https://tinyurl.com/rdjd57hh\\
% The program uses an eye detection to alert the driver if they are drowsy thereby preventing fatal crashes.\\
% \vspace*{2mm}
% {\textbf{Personal Website}} {\sl HTML5, CSS3, Javascript} \hfill https://pandyah5.github.io/\\
% A more detailed and informal description of who I am. Built using raw HTML, CSS and PHP from scratch.\\
% \vspace*{1mm}
{\textbf{Magnum Opus}} {\sl Python, Neural Style Transfer, OpenCV} $\cdot$ \href{https://pandyah5.github.io/magnum_opus/pista.html}{View Project} \\
My personal journey of finding “art in mathematics” and “mathematics in art”.\\
\vspace*{2mm}
% {\textbf{My Other Projects}}\hfill https://tinyurl.com/wd2c893h\\
% You can explore my other projects on Computer Vision, Digital Art, Data Analysis, Web Automation and Machine Learning.\\
% \vspace*{2mm}

%==== Extra Curriculars ====%
\header{Extra Curriculars}
\vspace{1mm}

\textbf{GitHub Education Program} \hfill Toronto, CA\\
\textit{GitHub Campus Expert} \hfill September 2022 - Present\\
\vspace{-3mm}
\begin{itemize} \itemsep 0.5pt
	\item Selected as one of the \textbf{65 global campus experts} in the 2022 cohort. Organized many open-source workshops and the first Github Field Day in Canada.
\end{itemize}

\textbf{UofT Machine Intelligence Student Team} \hfill Toronto, CA\\
\textit{Co-President} \hfill July 2021 - July 2022\\
\vspace{-3mm}
\begin{itemize} \itemsep 0.5pt
	\item Managed a club with \textbf{2000+ active members} and collaborated with different organizations like the Eng. Hatchery, UCL AI Society, AI@MIT, Harvard Open Data Project and many more.
\end{itemize}

\textbf{UofT Engineering Society} \hfill Toronto, CA\\
\textit{ECE Board of Director Representative} \hfill April 2022 - April 2023\\
\vspace{-3pt}
\begin{itemize}
	\item Elected to represent \textbf{700+ students} in the student-run UofT Engineering Society. I collaborated with other representatives to make executive decisions that offers services to \textbf{6000+ students}.
\end{itemize}

\header{Awards}
\textbf{AWS Student Life Hacks - First Prize} \hfill March 2024\\
% Recognized for the PEYDoor project built during the Hackathon. \hfill March 2024\\
\vspace*{2mm}
\textbf{Moral Code Hackathon - Third Prize} \hfill March 2022\\
% Recognized for AI Ethics project on the ethics of autonomous driving. \hfill March 2022\\
\vspace*{2mm}
\textbf{Microsoft Discover AI Challenge on AI Ethics - First Prize} \hfill June 2021\\
% Recognized for the AI Ethics Pipeline built during the Hackathon. \hfill June 2021\\
\vspace*{2mm}
% \textbf{University of Toronto Dean’s Honor Award} \hfill University of Toronto\\
% Awarded for my consistent academic standing above 3.5 GPA in all semesters. \hfill May 2021\\
% \vspace*{2mm}
\textbf{UofT Hacks VIII - Second Prize} \hfill Feb 2021\\
% Recognized for E-Motion - computer vision enabled remote monitoring and
% control. \hfill Feb 2021\\
\vspace*{2mm}
\textbf{Edward S. Rogers Dept. of Computer Eng. Top Student Award} \hfill Sept 2020\\
% Awarded to top three students in the Department of Electrical and Computer
% Engineering. \hfill Sept 2020\\
\vspace*{2mm}
\textbf{Wallberg Undergraduate Scholarship Award} \hfill Sept 2020\\
% Awarded to top four students in UofT Engineering, based on academic
% performance. \hfill Sept 2020\\
\vspace*{2mm}
\textbf{Hack The Virus Hackathon Winner} \hfill Aug 2020\\
% Recognized for my project COVID-InfoBot based on speech controlled information system. \hfill Aug 2020\\
\vspace*{2mm}
\textbf{UofT NSBE Hacks - Second Prize} \hfill Feb 2020\\
% Recognized for my project Hands2Ears, real time ASL to speech conversion. \hfill Feb 2020\\
\vspace*{2mm}
\textbf{Bloomberg Hack Winner} \hfill Feb 2020\\
% Recognized for my project Hands2Ears in \textquotedbl{}First Time Hack\textquotedbl{} category at NSBE Hacks. \hfill Feb 2020\\
\vspace*{2mm}
\textbf{University of Toronto International Scholar’s Award Scholarship} \hfill May 2019\\
% Awarded to students for excellence in academics and a wide range of extracurriculars. \hfill May 2019\\

\end{document}
